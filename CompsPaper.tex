\documentclass[12pt,twocolumn]{article} 

\usepackage{oxycomps} % use the main oxycomps style file

\bibliography{references}

\pdfinfo{
    /Title (Literature Review)
    /Author (Eddie Valdez)
}

\title{Oxy CS Comps Proposal}

\author{Eddie Valdez}
\affiliation{Occidental College}
\email{evaldez@oxy.edu}

\begin{document}

\maketitle

\begin{abstract}
    This document serves as the a proposal for my Oxy CS Comps paper. My Oxy Comps project will consist of an application that I develop that will help users integrate mindfulness into their daily lives. 
\end{abstract}

\section{Problem Context and Technical Background}


The main reason why I want to develop this application is because I believe practicing mindfulness is essential to living less stressful lives by changing how we react and understand our emotions. According to UC Berkely, "mindfulness means maintaining a moment-by-moment awareness of our thoughts, feelings, bodily sensations, and surrounding environment, through a gentle, nurturing lens"\cite{greatergood}. Mindfulness also involves acceptance of everything that appears in consciousness, meaning that we pay attention to our thoughts and feelings without judging them—without believing, for instance, that there’s a “right” or “wrong” way to think or feel in a given moment. When we practice mindfulness, our thoughts tune into what we’re sensing in the present moment rather than rehashing the past or imagining the future. There are many approaches to practicing mindfulness but they all revolve around being fully present in the moment. Some techniques that have been shown to be useful include mindfulness of breathing where we use the breath as an anchor to the present moment, compassion-focused meditation where we provoke love and kindness towards ourselves and others to be in the present moment, and body scans which is when we are aware of different part of our bodies to anchor ourselves in the present moment \cite{behan_2020}. Another popular practice of mindfulness is walking mindfulness where the entire focus is on awareness of our feet in contact with ground to be in the present moment. It should be noted what the difference between mediation and mindfulness is. As stated before, mindfulness is being aware fully of the present in various forms and at any time. This means one can focus their attention to anything. Mediation is similar except that when one practices mediation, they usually set apart a certain time specifically to meditate and focus on doing nothing.

There continues to be ever growing research that shows the numerous benefits that mindfulness has on our health and overall well-being \citetitle{Keng2011Effects}. This is important because now more than ever people are experiencing higher rates of stress, burnout, anxiety, and depression especially because of the COVID-19 Pandemic. Most people would like to experience a life without all these mental phenomenon but it seems impossible not to. While this is true, many people do not realize that there is an alternative approach which is to be able to control how we react and engage with these negative mental phenomenon. This could potentially be done through being mindful. 

Many people try to be mindful, however usually people typically only practice it for a short period of time and then forget to keep practicing. This irregular practice of mindfulness makes it hard to be truly mindful in our daily lives. This is why I believe a mindfulness application that intentionally attempts to be regularly integrated into people's daily lives could be really effective in building the habit of mindfulness. I personally have experienced the negative effects of stress, anxiety, and depression and I found that a  daily integration of mindfulness in my life keeps these mental phenomenons from negatively affecting my life. When I first began my mindfulness journey, I found it really hard to ever practice being mindful. I would simply get distracted by everyday tasks and rarely ever remembered to engage in mindfulness practices like mediation, gratitude, or nature walks. It was not until I started using mindfulness apps that I began seeing the benefits of mindfulness. However, most mindfulness apps are limited to solely meditation and while that was a good start, I believe I was not fully engaged with mindfulness practices until I also started to practice mindfulness more holistically on my own. This includes practicing gratitude, spending time in nature, regular meditative reflections throughout the day, and general mindfulness reminders. I want these aspects of mindfulness that really helped me and have been shown to be beneficial to others to be key features in the application that I develop. These are the features that I want my application to have because I believe they can help more people become more mindful by making it easier to engage in Mindfulness practices holistically. 

The biggest challenges I think I will have to tackle are making an application that is well organized and designed so that I can be as simple as possible for a user to engage in evidence based mindfulness practices and get them to want to keep using the application. I think the design and content are very important because I want the application to be appealing to as many people as possible. Because of this, I want to get input from people who practice mindfulness or want to practice mindfulness. I would also like input from people who struggle with mental health with what they think would be useful in an app. This means it should have a variety of features to practice mindfulness and they should be well organized. It should be an enjoyable experience overall which is why it should be easy and appealing to navigate the different mindfulness practice tools.


I will develop my application using SwiftUI in Xcode. I want to use Swift because many people have iPhone's, iPad's, or iPod's and I want the application to be able to be used by as many people as possible.

\section{Prior Work}



Recently with the rise in awareness of the importance of mental health, many applications have been created to promote mindfulness. Some of the most popular apps that are related to what I want to develop are Headspace, Calm, Breethe, Reflectly, and Waking Up. Most of these apps however, are essentially just mediation timers, guided meditations, and reminders to meditate. Apps that consist of just these features have shown very little evidence for their efficacy in developing mindfulness. They all mainly revolve around meditation and that is where I want my app to stand out. I want my application to include all aspects of mindfulness like a section for you to spend time in nature through a map with different locations, an information section where you can read about what mindfulness even is and how to practice it, reminders throughout the day regarding mindfulness tips like to slow down or focusing on one thing at a time along with wise inspirational quotes, and also a simple relaxing timer for you to be able to meditate easily along with prompts to reflect on. 

Every application mentioned above is very difficult to navigate because there is so much content. With that means the apps aren't very enjoyable to navigate which negatively effects the user experience with the app. I want my app to only have a 3 different sections with not an overwhelming amount of content. Because of this, I think my app will be a lot more visually appealing. A key difference in why there will be a lot less content is because I will not have guided meditations be the focus of my app. This is the main reason why Headspace, Calm, Breethe, and Waking Up are not easy to navigate: because they have an overwhelming variety of Guided Meditations that are all showcased for the user to select the best one for them. I believe this is ineffective as it never made me want to use them. There is also evidence that this makes these apps less effective \cite{Eysenbach2015Review}. An application that has a simpler user interface is Reflectly. This application is mostly an AI journal but its design is aesthetically pleasing and it doesn't include guided meditations. I also like that this app has daily positive quotes along with sending short helpful messages throughout the day so I want to implement similar features in my application.

Many apps have different aspects of what I want to create but none of them encompass all my desired features. Importantly, non of them implement a feature that includes locations where one can go to spend time in nature and be mindful there. I think this is an essential feature because it can be difficult to find these kinds of locations on your own and spending time in nature is an important part of developing mindfulness. I believe that these changes can help a user of my application to develop mindfulness because they are the things that helped me develop mindfulness for myself and there is evidence that some of these features I am wanting to implement can be effective in developing mindfulness.

\section{Methods and Evaluations}

The method that I will develop my app will be mostly research based and community input based. I will first brainstorm some ideas and research those ideas in order to create potential ideas of the components of my app that are based on evidence. I will then consult with other people who would potentially use my app to consider their opinion on how the app should be designed before I begin the wire frame of my app. I want to get the opinion of family members, friends, and other people who may use the app. I want to have interviews with people who practice mindfulness, new mindfulness people, and people who are not interested in mindfulness. I think everyone's opinion would be useful. I also want to get the opinion of low in come people and if possible, unaccompanied migrant youth. I think it would be useful to get community input because as Shasha Costaza-Chock says, it is important to have the community you are designing for be a part of the design process. This is why I am not solely planning the app out on my own. After I get community input, I want to start the design of different features of my app.I will then create a wire frame using Adobe Photoshop. I want to then start developing my app using SwiftUI and test the app for functionality. I will then have some people use the app and give me their opinion. After listening to what they have to say I will make changes accordingly. I want to do this several times. The people that will participate in the testing will most likely be strangers that I find online through surveys or forums.


For the evaluation of my app, I am considering multiple methods of evaluations. It has been difficult to decide which method of evaluation is I could use given the limited range of resources that I have. I have come to the conclusion that I will use the Five Facet Mindfulness Questionnaire. The Five Facet Mindfulness Questionnaire is a psychological measurement that tests how mindful a person is. The Five Facet Mindfulness Questionnaire is based on five independent aspects of mindfulness in the form of a questionnaires. The questionnaire consists of 39 items. This questionnaire has been shown to be effective in assessing how mindful someone is \cite{choi26}. Because of this, I will use this method of evaluation to evaluate if my app helped promote mindfulness. I will give out this survey before the the users use my app and then a few weeks after using my app to determine if their Five Facet Mindfulness Questionnaire score increases. 

I considered using other methods of evaluation but they are mostly all not feasible.Mindfulness practices that have been has been shown to improve aspects of psycho-social well-being through in-person training programs such as mindfulness-based stress reduction (MBSR) and mindfulness-based cognitive therapy (MBCT) have used randomized controlled trails that take months or years even to measure effectiveness. A lot less is known about the efficacy of digital training mediums, such as smartphone apps a big part is the lack of methods to test the efficacy. Some apps used controlled trails to test their apps by  assigning novice meditators to be randomly allocated to a mindfulness  application or to something like a  psycho-educational audio book control that feature an introduction to the concepts of mindfulness and meditation. The interventions have usually been be delivered via the same mindfulness app. At the end, evaluations of improvements in mental health aspects are usually  measured. This would be great but the scope is too large for me to do this.

Another method of evaluation that I considered that has been shown to be effective in determining if people are any more mindful. This is done through directly mean=sing the electrodes in peoples head's to measure pre and post therapy change in the electrophysiological brain response during mindfulness meditation and the long-term therapy outcome. This method has been shown to be very accurate, however it is not feasible given my limited resources. 

\section{Ethical Considerations}
Ethical issues in data bias could exist. I am a straight cisgender middle class male Latino Computer Science major who was raised Catholic and am attending a prestigious private liberal arts college. I am not an expert by any means on cognitive science, psychology, behavior, or mindfulness practices broadly. This could lead to false information being incorporated into my app. I am aware of my privileges as a male cisgender identity as well as my ethnicity as a Latino person. Because of this, I will try to be actively aware of designing my app to be as unbiased as possible. However, there are definitely  biases that I have that I am not consciously aware of. Because I am the only person who will be designing the technical aspects of this app, this could lead to my own biases being coded into the design of my app. However, in order to minimize this, I plan on having input from community members about the design and functionality of my app.

To this point, Sasha Costanza-Chock in her book design justice talks about how when we design products, we must design them with the community it is intended to help in order to meet their needs and remove our own biases\cite{Chock20}. Chock notes that most designers "do not think of themselves as sexist, racist, homophobic, etc..", however, their own perspectives may inhibit their ability to see how their design choices negatively affect oppressed communities\cite{Chock20}. This means including community members, specifically people who have been oppressed, in the design process and giving them credit for it. I have not included community input in planning the design of my app, however I plan on doing it to help my app be more useful and rid of biases.

My app would have multiple ethical accessibility issues. The first accessibility issue would be that because it will be an iOS app, only people with iPhones, iPads, or iPods would be able to use the app. This means that there are significant accessibility issues as many people do not have these types of devices. This might further perpetuate existing mental health inequalities because people without phones, meaning they are likely of lower socioeconomic status,have a higher likelihood of developing and experiencing mental health problems\cite{WHO14}. This is an issue because this is the audience that I would want my app to help the most. Additionally, The fact that my app will be in English will limit who can use my app. Only those who can read English would be able to find the app potentially useful. However, I do plan on making my app accessible using iOS accessibility features like VoiceOver using text descriptions of my app to make my app more accessible to people with visual impairments, dyslexia, and other challenges to navigate the Apple device interface using gestures. 


Another accessibility issue includes the fact that my app would not likely be effective for unaccompanied migrant youth. This is because unaccompanied migrant youth themselves are not going to be co-designers of my app. I conclude this because research shows that when unaccompanied migrant youth are not co-designers of mental health technologies and the macro system’s influence is not considered in the design, mental health technologies have been shown to be ineffective\cite{Tachtler21}. Marco system influences include cultural, linguistic and socio-economical factors. So because I am not having unaccompanied migrant youth be a part of my design process, it may not be effective for them. This research study supports Sasha Costanza Chock’s point that the community we are helping should be a part of the design process\cite{Chock20}. Because of this, I will try to find unaccompanied migrant youth and get their input on my app.

My app also faces the ethical question of whether making an app to help people be mindful is an effective solution to improving people’s mental health. This is important to consider because maybe an app is not the proper solution to helping people be mindful. In fact there is a significant amount of research that shows that there is little evidence is available on the efficacy of apps in developing mindfulness. Many apps exist that claim to be mindfulness-related. However, most are guided meditation apps, timers, or reminders and very few have high ratings on scores of visual aesthetics, engagement, functionality or information quality\cite{Mani2015ReviewAE}. Additionally, it is also important to consider whether or not promoting mindfulness and mediation is ethical. This is because there is evidence that too much mindfulness can cause harm to one's mental health\cite{Britton2019CanMB}.There is also evidence that suggests that too much meditation can have negative undesired consequences to one's mental health \cite{Cebolla2017}. There is not too much evidence of the negative effects of mindfulness but that is because there is lack of research into the full range of possible effects of mindfulness, not just positive effects. Researchers agree, that more diverse research into the negative effects of mindfulness is necessary \cite{Britton2019CanMB}.  Despite this, there is some evidence that practicing mindfulness using a smartphone app may provide immediate positive effects on mood and stress while also providing long-term benefits for attention control. Because of this I am hopeful that a well designed mindfulness app could be an effective technology solution to develop mindfulness\cite{Walsh2019EffectsOA}.

One last but important ethical would be the issue of user privacy. Many mindfulness/mediation apps have gotten much criticism because of their violations of user privacy including selling valuable user to advertising companies but only a very small percentage of them disclosed this \cite{Huckvale2019}. In my app, I want to no user data to be shared or sent to advertisers. All user data will be stored locally on their device and I will try to design the app so third parties can not access this information. 
\section{Timeline}
\begin{itemize}
  \item 5/15: Start brainstorming ideas of features of the app and approaches in how those apps will be implemented.
  \item 6/1: Vacation
  \item 6/15: Begin interviewing people, posting questionnaire online, asking people in general their opinions of my app ideas.
  \item 7/1: Determine if you have enough feedback. If not use other forms of media to gain more feedback on the design of my app. 
  \item 7/15: Use feedback to start creating a wire frame of your app design.
  \item 8/1: Vacation
  \item 8/15: Show people your design and get feedback. 
  \item 9/1: Use feedback to finalize the design of your app. Begin developing my app.
  \item 9/15: Continue building my app. Engage in frequent testing for functionality.
  \item 10/1: Finalize the development of your app.
  \item 10/15: Find people who will test your app to determine if it is effective in improving mindfulness.
  \item 11/1: People are now testing your app
  \item 11/15: Use the The Five Facet Mindfulness Questionnaire to determine if you app was effective.
  \item 12/1: Finalize results and CS Comps Paper.
  \item 12/5: Present Comps Project
  
\end{itemize}
\printbibliography 

\end{document}
